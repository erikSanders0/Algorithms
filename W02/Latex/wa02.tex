
\documentclass[12pt]{amsart}
\usepackage{geometry} % see geometry.pdf on how to lay out the page. There's lots.
\geometry{a4paper} % or letter or a5paper or ... etc
\usepackage[T1]{fontenc}
\usepackage[latin9]{inputenc}
\usepackage{amsmath}
\usepackage{amsaddr}
\usepackage{hyperref}
\usepackage{dirtytalk}
\usepackage{float}
\usepackage{listings}
\usepackage{hyperref}
\usepackage{enumerate}

\usepackage{color}
 
\definecolor{codegreen}{rgb}{0,0.6,0}
\definecolor{codegray}{rgb}{0.5,0.5,0.5}
\definecolor{stringcolor}{rgb}{0.7,0.23,0.36}
\definecolor{backcolour}{rgb}{0.95,0.95,0.92}
\definecolor{keycolor}{rgb}{0.007,0.01,1.0}
\definecolor{itemcolor}{rgb}{0.01,0.0,0.49}

\hypersetup{
    colorlinks=true,
    linkcolor=blue,
    filecolor=blue,      
    urlcolor=blue,
}
 
\lstdefinestyle{mystyle}{
    %backgroundcolor=\color{backcolour},   
    commentstyle=\color{codegreen},
    keywordstyle=\color{keycolor},
    numberstyle=\tiny\color{codegray},
    stringstyle=\color{stringcolor},
    basicstyle=\footnotesize,
    breakatwhitespace=false,         
    breaklines=true,                 
    captionpos=b,                    
    keepspaces=true,                 
    numbers=left,                    
    numbersep=5pt,                  
    showspaces=false,                
    showstringspaces=false,
    showtabs=false,                  
    tabsize=2
}
 
\lstset{style=mystyle}

\lstdefinelanguage{Swift}{
  keywords={associatedtype, class, deinit, enum, extension, func, import, init, inout, internal, let, operator, private, protocol, public, static, struct, subscript, typealias, var, break, case, continue, default, defer, do, else, fallthrough, for, guard, if, in, repeat, return, switch, where, while, as, catch, dynamicType, false, is, nil, rethrows, super, self, Self, throw, throws, true, try, associativity, convenience, dynamic, didSet, final, get, infix, indirect, lazy, left, mutating, none, nonmutating, optional, override, postfix, precedence, prefix, Protocol, required, right, set, Type, unowned, weak, willSet},
  ndkeywords={class, export, boolean, throw, implements, import, this},
  sensitive=false,
  comment=[l]{//},
  morecomment=[s]{/*}{*/},
  morestring=[b]',
  morestring=[b]"
}

\lstset{emph={Int,count,abs,repeating,Array}, emphstyle=\color{itemcolor}}


\title{Week 02}

\date{\today}

\lstset{style=mystyle}

%%% BEGIN DOCUMENT
\begin{document}
\maketitle
\section{Preparation for Assignment}
If, and \textit{only if} you can truthfully assert the truthfulness of each statement below are you ready to start the assignment.
\subsection {Reading Comprehension Self-Check}
\begin{itemize}
	\item I know that an algorithm's time efficiency is principally measured as a
    function of its input size by counting the number of times its basic
    operation is executed.
	\item  I understand why it is \textbf{false} to say that a \textbf{basic operation} of an
    algorithm is the operation that contributes \textbf{least} toward the algorithm's
    running time.
	\item I understand that the established framework for analyzing an algorithm's
    time efficiency is primarily grounded in the order of growth of the
    algorithm's running time as its input size goes to infinity.
	\item I have pondered the idea that the main tools for analyzing the time
    efficiency of a non-recursive algorithm are to set up a sum expressing the
    number of executions of its basic operations and then to ascertain the sum's
    order of growth.
	 \item I understand that there are several algorithms for computing the
    Fibonacci numbers with drastically different efficiencies.
  \item I will ponder the idea that the main tools for analyzing the time
    efficiency of a recursive algorithm are to set up a recurrence relation
    expressing the number of executions of its basic operations and then to
    ascertain the solution's order of growth.
  \item I know that \textit{empirical analysis} is applicable to any algorithm.
  \item I know why it is \textit{false} to say that \textit{Algorithm Visualization}, or the
    use of images to convey useful information about algorithms, has three
    principal variations.
  \item I discovered a web site devoted to algorithm visualization/animation.

\end{itemize}















\subsection{Memory Self-Check}

\subsubsection{Statements Regarding Algorithms}
Rank each quote in the About Algorithms reading. Rank them by your preference for them. Explain why your top and bottom ranked quotes were ranked as first and last.
\subsubsection{Pondering Brings Depth}
Ponder on the methodology you used rank the quotes. Convert the methodology into an algorithm. What is the $\mathcal{O}$ order of growth for your algorithm?

\subsubsection{Fill in the Blank}

\[ \lim_{n \rightarrow \infty} \frac{t(n)}{g(n)} = \begin{cases}
    0\quad \text{the order of growth of t(n) less than the order of growth of g(n)}\\
    c>0\quad\text{the order of growth of t(n) equal the order of growth of g(n)}\\
    \infty \quad  \text{the order of growth of t(n) greater than the order of growth of g(n)}
\end{cases} 
\]

\subsubsection{Apply Limits to Specific Function Pairs}
Calculate $\lim_{n \rightarrow \infty} \frac{t(n)}{g(n)}$ for each pair of functions.  

\[\begin{array}{lcl}
     t(n) & g(n) & \lim_{n \rightarrow \infty} \frac{t(n)}{g(n)}\\ \hline
     10n   & 2n^2 & 0\\
     n(n+1)/2  & n^2 & 1 \\
     \log_{b}n  & \log_{c}n & 1
     \end{array}\]

Hint: L'H\^{o}pital's rule says:

   \[\lim_{n \rightarrow \infty} \frac{t(n)}{g(n)} = \lim_{n \rightarrow \infty} \frac{t^{\prime}(n)}{g^{\prime}(n)}\]
















\section{Week 2 Exercises}
\subsection{ Exercise 1 on page 50} $ $\\ 

Here is an example of a good answer for algorithm a:
  
  \begin{center}
\begin{tabular}{ |c|c|c|c| } 
 \hline
 \textbf{Algorithm} & \textbf{i} & \textbf{ii} & \textbf{iii} \\ 
 \hline
 $\sum{n}$ & n & addition of two numbers & no \\ 
 \hline
 $n!$ & n &multiplication of two numbers & no\\ 
 \hline
 $largestN$ & n & comparison between two elements & yes \\ 
 \hline
Euclids & n & comparison between n and 0& no \\
 \hline
Sieve of Eratosthenes & n & comparison betwwen j and n & no \\  
 \hline
\end{tabular}
\end{center}

\subsection{Exercise 8 on page 51} $ $\\ 

  Here is an example of a good answer for function a:
  If $x$ changes from $1n$ to $4n$, the value of $\log_2 x$ increases by $2$.
  Proof:\\
a.
 \begin{align*}  
 \log_2 4n - \log_2 n &= (\log_2 4 + \log_2 n) - \log_2 n \\
 		&= 2 + \log_2 n - \log_2 n \\
 		&= 2
\end{align*}
b.
\begin{align*}
\sqrt{n4} - \sqrt{n} &= (\sqrt{n} + 2) - \sqrt{n} \\
& = 2 + \sqrt{n} - \sqrt{n} \\
& = 2 \\
\end{align*}
c.
\begin{align*}
4n - n &= n(4 - 1) \\
& = 3n \\
& = 3 \\
\end{align*}
d.
\begin{align*}
4n^2 - n^2 &= n^2(4 - 1) \\
& = 3n^2 \\
& = 3 \\
\end{align*}
e.
\begin{align*}
4n^3 - n^3 &= n^3(4 - 1) \\
& = 3n^3 \\
& = 3 \\
\end{align*}
f.
\begin{align*}
4(2^n) - 2^n &= 2^n(4 - 1) \\
& =  2^n (3)\\
& = 3 \times 2^n\\
\end{align*}

\subsection{Exercise 3 on page 59} $ $\\ 

a.
\[ \lim_{n \rightarrow \infty} \frac{(n^2 + 1)^{10}}{n^{20}} =
   \lim_{n \rightarrow \infty} \frac{(n^2 + 1)^{10}}{(n^2)^{10}} =
   \lim_{n \rightarrow \infty} \left( \frac{n^2 + 1}{n^2} \right)^{10} =
   \lim_{n \rightarrow \infty} \left( 1 + \frac{1}{n^2} \right)^{10} = 1.
\]
$(n^2 + 1)^{10} \approx (n^2)^{10} = n^{20} \in \Theta(n^{20})$\\

b. 
\[
\lim_{n \rightarrow \infty} \frac{\sqrt{10n^2 + 7n + 3}}{n} = \sqrt{10}\\
\]
$\sqrt(10n^2 + 7n + 3) \approx \sqrt{n^2} = n \in \Theta(n)\\$\\

c. 
\[
\lim_{n \rightarrow \infty} \frac{2n \log{(n+2)^2}+(n+2)^2 \log{\frac{n}{2}}}{n^2\log{n}} = 1\\
\]
$2n \log{(n+2)^2}+(n+2)^2 \log{\frac{n}{2}} \approx n^2 \log{n}  \in \Theta(n^2 \log{n})\\$\\

d.
\[
\lim_{n \rightarrow \infty} \frac{2^{n+1} + 3^{n-1}}{3^n} = \frac{1}{3}\\
\]
$2^{n+1} + 3^{n-1} \approx 3^{n-1}  = 3^n \in \Theta(3^n)\\$\\

e. $\log{n}$
\[
\lim_{n \rightarrow \infty} \frac{\lfloor{\log_2{n}}\rfloor}{\log{n}} = \frac{1}{\log{2}}\\
\]
$\lfloor{\log_2{n}}\rfloor \approx \log_2{n}  = \log{n} \in \Theta(\log{n})\\$\\


\subsection{Exercise 1 on page 67} $ $\\ 

a. 
  \[1 + 3 + 5 + 7 + \ldots + 999 = \sum_{i = 1}^{500} (2i - 1) = \sum_{i = 1}^{500} 2i - \sum_{i = 1}^{500} 1
= 2 \frac{500 * 501}{2} - 500 = 250,000.\]

  $a_1 = 1, a_n = 999, n = 500$, 
  \[\frac{(a_1 + a_n)n}{2} = \frac{(1+999)500}{2} = 250,000.\]\\


b.
  $a_1 = 2, a_n = 1024, n = 512$, 
  \[\frac{(a_1 + a_n)n}{2} = \frac{(2+1024)512}{2} = 262,656.\]\\

c.
 \[ \sum_{i=3}^{n+1} 1 = n \] \\

d.
 \[ \sum_{i=3}^{n+1} i = n^2 \]\\
  $a_1 = 2, a_n = 1024, n = 512$, 
  \[\frac{(a_1 + a_n)n}{2} = \frac{(3+n)n}{2} = \frac{3n}{2} + \frac{n^2}{2} = n^2 .\]\\

e.
 \[ \sum_{i=0}^{n-1} i(i+2) = n^3 \]\\

f.
 \[ \sum_{j=1}^{n} 3^{j+1} = 3^n \]\\

g.
 \[ \sum_{i=1}^{n} \sum_{j=1}^{n}ij = n^2 \]\\

h.
 \[ \sum_{i=1}^{n} \frac{1}{i(i+1)} = n^2 ??? \]\\

\subsection{Exercise 2 on page 67} $ $\\ 


a.
\[ \sum_{i = 0}^{n - 1} (i^2 + 1)^2 = \sum_{i = 0}^{n - 1}(i^4 + 2i^2 + 1) = \sum_{i = 0}^{n - 1} i^4 + 2 \sum_{i = 0}^{n - 1} i^2 + \sum_{i = 0}^{n - 1} 1 \in \Theta(n^5) + \Theta(n^3) + \Theta(n) = \Theta(n^5)
\]
\[.
\]

b.
\[ \sum_{i = 2}^{n - 1}\log{i^2} = 2 \sum_{i = 2}^{n - 1} \log{i} \in \Theta(n \log{n})) = \Theta(n \log{n})
\]
\[.
\]

c.
\[ \sum_{i = 1}^{n} (i + 1)2^{i-1} = \sum_{i = 1}^{n} i2^{i-1} + 2^{i-1} =  \sum_{i = 1}^{n}  i2^{i-1} + \sum_{i = 1}^{n} 2^{i-1}  \in \Theta(n2^n) + \Theta(2^n) = \Theta(n2^n)
\]
\[.
\]

d.
\[ \sum_{i = 0}^{n - 1} \sum_{j = 0}^{i - 1}  (i+j) = \sum_{i = 0}^{n - 1} \sum_{j = 0}^{i - 1}i + \sum_{i = 0}^{n - 1}  \sum_{j = 0}^{i - 1}j \in \Theta(n^2) + \Theta(n^2) = \Theta(n^2)
\]
\[.
\]

\subsection{Exercise 3 on page 76} $  $\\

a. It appears that the if statement is the basic operation in this function.  \\
b. It is equivalent to a nonrecursive function.\\
   
\section{Week 2 Problems}

\subsection{Exercise 12 on page 61}

\subsection{Analysis Using $\mathcal{O}(f(n))$}
   If it wasn't already part of your comparison/contrast of the three algorithms
  you created for the "Breaking Up" problem of last week's assignment, do a
  $\mathcal{O}(f(n))$ analysis of each algorithm here.



\subsection{Exercise 6 on page 83} $ $\\
a. 49 \\
b. 94

\subsection{Exercise 9 on page 84} $ $\\ 

   \textit{Hint: Use mathematical induction.}
\end{document}